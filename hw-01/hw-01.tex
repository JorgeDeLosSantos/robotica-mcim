\documentclass[12pt,letterpaper]{article}
\usepackage[utf8]{inputenc}
\usepackage[spanish,es-tabla]{babel}
\decimalpoint
\let\cleardoublepage\clearpage
\usepackage{amsmath}
\usepackage{amsfonts}
\usepackage{amssymb}
\usepackage{esint}
\usepackage{color}
\usepackage{graphicx}
\usepackage{anysize}
\usepackage{anyfontsize}
\usepackage{pdfpages}
\usepackage{makeidx}
% \usepackage{mathtools}
\makeindex 
\usepackage[x11names,table]{xcolor}
\usepackage{tikz}
\usepackage{tcolorbox}
\usepackage[hidelinks]{hyperref}
\usepackage{caption}
\usepackage{listings}
\usepackage{bm}
\usepackage[left=2cm,top=2cm,right=2cm,bottom=2cm]{geometry}
\setlength{\parindent}{0cm}
\tcbset{colback=green!5!white, colframe=gray!10!black, coltitle=green!20!black, 
fonttitle=\bfseries, colbacktitle=white, coltext=gray!30!black}
\addto\captionsspanish{
  \renewcommand{\figurename}{{\bf Figura}}% 
}
\usepackage{epigraph}
\usepackage{xcolor}

% Colores
\definecolor{verdep}{rgb}{0.5,0.5,0.9}
\definecolor{ccap}{rgb}{0.2,0.2,0.2}
\definecolor{csec}{rgb}{0.4,0.4,0.4}
\definecolor{csubsec}{rgb}{0.6,0.6,0.6}
\definecolor{cenun}{rgb}{0.2,0.2,0.3}
\definecolor{csol}{rgb}{0.2,0.8,0.1}
\definecolor{backcode}{rgb}{0.95,0.95,0.99}
\definecolor{dkgreen}{rgb}{0,0.6,0}
\definecolor{gray}{rgb}{0.5,0.5,0.5}
\definecolor{mauve}{rgb}{0.58,0,0.82}

% Nuevos comandos

\usepackage{titlesec}%--
\newcommand{\hsp}{\hspace{5pt}}

% Code

\lstnewenvironment{apdl}{\lstset{frame=single,
  frameround=tttt,
  backgroundcolor=\color{backcode},
  language={},
  aboveskip=3mm,
  belowskip=3mm,
  showstringspaces=false,
  columns=flexible,
  basicstyle={\small\ttfamily},
  numbers=none,
  numberstyle=\tiny\color{gray},
  keywordstyle=\color{blue},
  commentstyle=\color{dkgreen},
  stringstyle=\color{mauve},
  breaklines=true,
  breakatwhitespace=true,
  tabsize=3,
  extendedchars=true,
  inputencoding=utf8,
  literate=%
  {°}{{\,\,$^\circ$\,\,}}1
  {á}{{\'a}}1
  {é}{{\'e}}1
  {í}{{\'i}}1
  {ó}{{\'o}}1
  {ú}{{\'u}}1
  {Á}{{\'A}}1
  {É}{{\'E}}1
  {Í}{{\'I}}1
  {Ó}{{\'O}}1
  {Ú}{{\'U}}1
}}{}


\author{Pedro Jorge De Los Santos Lara}
\title{Robótica - MCIM}


% ======================================================================================================
\begin{document}
\maketitle

\textbf{1.} Probar que: $ ^{j}\vec{\omega}^{m} \times \vec{b_{2}} = \frac{^{j} d \vec{b_2} }{dt}  $. \\

Con la expresión de la velocidad angular $^{j}\vec{\omega}^{m}$ podemos escribir:

$$ 
^{j}\vec{\omega}^{m} \times \vec{b_{2}} = 
\left( 
\vec{b_1} \frac{^{j} d \vec{b_2} }{dt} \cdot \vec{b_3} +
\vec{b_2} \frac{^{j} d \vec{b_3} }{dt} \cdot \vec{b_1} +
\vec{b_3} \frac{^{j} d \vec{b_1} }{dt} \cdot \vec{b_2}
\right) \times \vec{b_2}
\,\, = \,\,
\vec{b_3} \frac{^{j} d \vec{b_2} }{dt} \cdot \vec{b_3} - 
\vec{b_1} \frac{^{j} d \vec{b_1} }{dt} \cdot \vec{b_2}
$$

Luego, se tiene que:

$$
\vec{b_2} \cdot \vec{b_2} = 1
$$

Entonces:

$$ 
\frac{^{j} d \vec{b_2} }{dt} \cdot \vec{b_2} + \vec{b_2} \cdot \frac{^{j} d \vec{b_2} }{dt} = 0
 \,\,\,\,\,\, \rightarrow \,\,\,\,\,\,
\frac{^{j} d \vec{b_2} }{dt} \cdot \vec{b_2} = 0  
\,\,\,\,\,\, \rightarrow \,\,\,\,\,\,
\vec{b_2} \frac{^{j} d \vec{b_2} }{dt} \cdot \vec{b_2} = \vec{0}
$$

De manera similar para:

$$
\vec{b_2} \cdot \vec{b_1} = 0
$$

Se tiene que:

$$
\frac{^{j} d \vec{b_2} }{dt} \cdot \vec{b_1} + \vec{b_2} \cdot \frac{^{j} d \vec{b_1} }{dt} = 0
 \,\,\,\,\,\, \rightarrow \,\,\,\,\,\,
\frac{^{j} d \vec{b_2} }{dt} \cdot \vec{b_1} = - \frac{^{j} d \vec{b_1} }{dt} \cdot \vec{b_2}
$$

Sustituyendo en la expresión obtenida:

$$
^{j}\vec{\omega}^{m} \times \vec{b_{2}} = 
\frac{^{j} d \vec{b_2} }{dt} \cdot \vec{b_1}  \vec{b_1} +
\frac{^{j} d \vec{b_2} }{dt} \cdot \vec{b_2}  \vec{b_2} +
\frac{^{j} d \vec{b_2} }{dt} \cdot \vec{b_3}  \vec{b_3} = 
\frac{^{j} d \vec{b_2} }{dt}
$$

\vspace{5mm}
\hline
\vspace{5mm}

\textbf{2.} Probar que: $ ^{j}\vec{\omega}^{m} \times \vec{b_{3}} = \frac{^{j} d \vec{b_3} }{dt}  $. \\

Con la expresión de la velocidad angular $^{j}\vec{\omega}^{m}$ podemos escribir:

$$ 
^{j}\vec{\omega}^{m} \times \vec{b_{3}} = 
\left( 
\vec{b_1} \frac{^{j} d \vec{b_2} }{dt} \cdot \vec{b_3} +
\vec{b_2} \frac{^{j} d \vec{b_3} }{dt} \cdot \vec{b_1} +
\vec{b_3} \frac{^{j} d \vec{b_1} }{dt} \cdot \vec{b_2}
\right) \times \vec{b_3}
\,\, = \,\,
- \vec{b_2} \frac{^{j} d \vec{b_2} }{dt} \cdot \vec{b_3} +
\vec{b_1} \frac{^{j} d \vec{b_3} }{dt} \cdot \vec{b_1}
$$

Luego, se tiene que:

$$
\vec{b_3} \cdot \vec{b_3} = 1
$$

Entonces:

$$ 
\frac{^{j} d \vec{b_3} }{dt} \cdot \vec{b_3} + \vec{b_3} \cdot \frac{^{j} d \vec{b_3} }{dt} = 0
 \,\,\,\,\,\, \rightarrow \,\,\,\,\,\,
\frac{^{j} d \vec{b_3} }{dt} \cdot \vec{b_3} = 0  
\,\,\,\,\,\, \rightarrow \,\,\,\,\,\,
\vec{b_3} \frac{^{j} d \vec{b_3} }{dt} \cdot \vec{b_3} = \vec{0}
$$

De manera similar para:

$$
\vec{b_2} \cdot \vec{b_3} = 0
$$

Se tiene que:

$$
\frac{^{j} d \vec{b_2} }{dt} \cdot \vec{b_3} + \vec{b_2} \cdot \frac{^{j} d \vec{b_3} }{dt} = 0
 \,\,\,\,\,\, \rightarrow \,\,\,\,\,\,
\frac{^{j} d \vec{b_3} }{dt} \cdot \vec{b_2} = - \frac{^{j} d \vec{b_2} }{dt} \cdot \vec{b_3}
$$

Sustituyendo en la expresión obtenida:

$$
^{j}\vec{\omega}^{m} \times \vec{b_{3}} = 
\frac{^{j} d \vec{b_2} }{dt} \cdot \vec{b_1}  \vec{b_1} +
\frac{^{j} d \vec{b_2} }{dt} \cdot \vec{b_2}  \vec{b_2} +
\frac{^{j} d \vec{b_2} }{dt} \cdot \vec{b_3}  \vec{b_3} = 
\frac{^{j} d \vec{b_3} }{dt}
$$

\end{document}